\documentclass[11pt,a4paper]{article}
\usepackage{acl2015}
\usepackage{times}
\usepackage{url}
\usepackage{latexsym}


\title{Data Mining en Social Media. M\'aster Big Data Analytics.}

\author{David Selma Herrero \\
  {\tt daselher@masters.upv.es} \\}

\date{}



\begin{document}
\maketitle
\begin{abstract}
  El reto que se nos plantea es predecir a que genero (hombre, mujer) o variedad (pa\'is) pertenecen los tweets. Para ello se nos facilitan dos carpetas, una para training y otra para test. En estas dos carpetas tenemos un fichero donde podemos identificar el autor (nombre del fichero XML), genero (si es hombre o mujer), y su variedad etiquetadas. Este fichero se llama "truth.txt". Y por ultimo el resto de ficheros XML, un fichero XML por autor.
  
  Para ello se realizan una serie de puntos en el siguiente orden:
  \begin{itemize}
  \item Entender los datos que manejamos y el problema al que nos enfrentamos.
  \item Estudio de la limpieza de datos, ya que dependiendo de que se limpie podemos mejorar o empeorar los resultados.
  \item Estudio y elaboraci\'on del algoritmo.
  \item Estudio de los modelos a aplicar a nuestro caso en particular.
  \item Uno de los par\'ametros a fijarnos principalmente: Accuracy y kappa.
  \end{itemize}
  
  Por ultimo, se han realizado diferentes pruebas con los diferentes modelos para obtener el mejor resultado en un tiempo razonable.
  
\end{abstract}


\section{Introducci\'on}

El objetivo de este problema es superar la predicci\'on base de genero y variedad.
  \begin{itemize}
  \item Genero: 66'43 \%
  \item Variedad: 77'21 \%
  \end{itemize}
  
  Se nos proporcionara un dataset para realizar una mejor predicci\'on. Y para ello habr\'a que superar dichos par\'ametros.
  
\section{Dataset}

  Se nos proporcionan un dataset ya separado en training y test, en dos carpetas diferentes. Cada uno de ellos contiene ficheros XML, uno por cada autor. Y un fichero llamado "truth.txt" donde tendr\'a nombre del autor, genero y variedad.

  La carpeta de training contiene un total de 2800 autores y la carpeta de test tiene un total de 1400 autores. Es decir, un 66'66 \% de training y un 33'34 \% de test.

  Cada autor tiene 100 tweets.

  Por una parte la carpeta de training contiene 1400 autores de genero masculino y 1400 autores de genero femenino. Por otra parte la carpeta de test contiene 700 autores de genero masculino y 700 autores de genero femenino.

  Se puede observar que este dataset de training esta compuesto por 400 autores de cada uno de los pa\'ises que tiene llamados Colombia, Argentina, Spain, Venezuela, Peru, Chile y Mexico. Tambi\'en el dataset de test tiene 200 autores de cada uno de los pa\'ises citados anteriormente.

\section{Propuesta del alumno}

  Para tener una predicci\'on mas exacta y precisa se propone primero una limpieza de datos y adaptaci\'on de la informaci\'on para este problema en los siguientes puntos:
  \begin{itemize}
  \item Transformar las palabras a min\'usculas (tolower).
  \item Eliminaci\'on de signos de puntuaci\'on (removePunctuation).
  \item Eliminaci\'on de n\'umeros (removeNumbers).
  \item Eliminaci\'on de palabras stopwords (removeWords de swlist = "es").
  \item Eliminaci\'on de preposiciones (removeWords).
  \item Eliminaci\'on de espacios en blanco extra. Varios caracteres de espacio en blanco se contraen a un solo espacio en blanco (stripWhitespace).
  \end{itemize}

Se realiza una bolsa de palabras adecuada para variedad y genero.

  Una vez se finalice la limpieza adecuada y elaborado el algoritmo que se considera que obtendr\'a una mayor precisi\'on en la predicci\'on. Se procede a aplicar los siguiente modelos que hemos considerado apropiados para este caso:
  \begin{itemize}
  \item Redes Neuronales
  \item Naive Bayes
  \item Random Forest
  \end{itemize}

En principio se entreno el modelo con una
m\'aquina de soporte de vectores con diferentes par\'ametros pero no hubo la mejora que se esperaba para la resolución de este problema.

De modo alternativo, cuando se probo con los modelos Redes Neuronales, Navive Bayes y Random Forest. Se observa que Random Forest daba mejores predicciones al incrementar el n\'umero de \'arboles a 100.

Para este problema se ha tenido en cuenta que no se este haciendo sobreajuste.

Por otra parte nos fijamos en Accuracy y en el Kappa. El Kappa es importante que este lo mas cercano a 0, ya que si no es asi significa que el resultado es altamente probable debido al azar.


\section{Resultados experimentales}

Los resultados experimentales obtenidos tras la realizaci\'on de este problema en las
predicciones de género y variedad son:
  \begin{itemize}
  \item Genero: 72'21 \%
  \item Variedad: 88'71 \%
  \end{itemize}

Estos resultados se obtienen al aplicar Random Forest con una configuraci\'on de 100 arboles. El c\'alculo se ha realizado dentro de un tiempo razonable y esperado.

Por falta de tiempo se procedi\'o a ejecutar Random Forest con diferentes arboles y esto llevo todo el d\'ia dando peores resultados a los anteriormente.


\section{Conclusiones y trabajo futuro}


  \begin{itemize}
  \item Conclusiones:
  Los resultados obtenidos han mejorado con respecto a la predicci\'on base dada inicialmente. La precisión de género ha mejorado en un 9'40 \% y la de variedad un 15'20 \%.
  

  \item Trabajo futuro:
  
  
    \begin{itemize}
    \item Realizaci\'on de la bolsa de vocabulario con las palabras mas utilizadas en cada uno de los pa\'ises, es decir, palabras propias de cada pa\'is o genero. Pudiendo as\'i identificar a que variedad pertenece. O en el caso del genero las palabras mas utilizadas por el genero femenino y tambi\'en las palabras mas utilizadas por el genero masculino.
    
    \item Longitud de las palabras ya que se piensa que un determinado genero o variedad escribe con mayor longitud los tweets.
    
    
    \item Sacar la ra\'iz o lexema de las palabras para predecir a que variedad pertenecen los tweets.
    
    
    \item Quitar la ra\'iz y quedarnos con las ultimas letras de las palabras para predecir a que genero pertenecen los tweets.    
    
    \item Leer tweets en grupos de dos o mas palabras detectar patrones que se puedan obtener en determinado genero o variedad.
    
    
    \item Dar peso a determinadas palabras. Esta idea viene porque se piensa que aquellas palabras mas frecuentes o propias de un determinado genero o variedad tiene que tener mas importancia y mayor impacto en la predicci\'on con respecto a aquellas palabras que son comunes en los dos g\'eneros o que est\'en en diferentes variedades. Las palabras comunes en ambos g\'eneros y diferentes variedades tendr\'ian que tener menor peso por su poca importancia en la predicci\'on.
    
    \end{itemize}
  
  \end{itemize}


\section{Referencias}

\url{http://topepo.github.io/caret/index.html}

\url{https://topepo.github.io/caret/available-models.html}


\end{document}
